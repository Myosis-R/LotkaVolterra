\documentclass{wsdcr}

\title{Étude du model Lotka-Volterra}
\author{Robin Botrel, Axel Carpentier}
\affil{\textit{Université Paul Sabatier}\\
\textit{Toulouse, France}}
\date{28 Decembre, 2022}
\bibliographystyle{IEEEtran}

\begin{document}

\maketitle
\tableofcontents
\section{Introduction}

\lettrine{C}{ette} étude des équations de prédation de Lotka-Volterra s'effectue dans le cadre d'une unité d'enseignement ouverte de la licence de mathématique de l'université Paul Sabatier. Nous allons partir d'un exercice vu en cours d'équations différentielles ordinaires (EDO), pour ensuite s'éloigner des frontières de ce cours et voir ce que peut offrir la discipline. 

\subsection{Le model classique proie prédateur}
\begin{equation}
\left\{
{\begin{array}{ccc}{\dfrac {\mathrm {d} x(t)}{\mathrm {d} t}}&=&x(t)\ {\Big (}\alpha -\beta y(t){\Big )}\\{\dfrac {\mathrm {d} y(t)}{\mathrm {d} t}}&=&y(t)\ {\Big (}\delta x(t)-\gamma {\Big )}\end{array}}
\right.
\end{equation}
\subsection{Généralisation}
\begin{equation}
\left\{
{\begin{array}{ccc}{\dfrac {\mathrm {d} x(t)}{\mathrm {d} t}}&=&x(t)\ {\Big (}a -b y(t)-c x(t){\Big )}\\{\dfrac {\mathrm {d} y(t)}{\mathrm {d} t}}&=&y(t)\ {\Big (}d x(t)-e -f y(t) {\Big )}\end{array}}
\right.
\end{equation}
\begin{equation}
\dfrac {\mathrm {d}}{\mathrm {d} t}}X(t)=X(t) {\Big (}R+AX(t){\Big )}
\end{equation}
\section{Le cas 2D}
\section{Chaos en 4D}
\begin{equation}
R={\begin{bmatrix}1\\0.72\\1.53\\1.27\end{bmatrix}}\quad A ={\begin{bmatrix}-1&-1.09&-1.52&0\\0&-0.72&-0.3168&-0.9792\\-3.5649&0&-1.53&-0.7191\\-1.5367&-0.6477&-0.4445&-1.27\end{bmatrix}}
\end{equation}
\section{Conclusion}
\section{Annexes}
\bibliography{rapport}

%An example of a figure is shown in Figure \ref{fig:example}. The recommended width of the figure is $0.9\times$\textbackslash linewidth, and the recommended width:height aspect ratio is $1.41:1$ (i.e. A4 paper)
%
%Additionally, one can create a minipage for extra information like biography as shown in the following:
%
%\footnotetext[1]{acquired from the official website of IPS at\\https://www.waseda.jp/fsci/gips/en/about/overview-2/}
%
%%\begin{figure}[t!]
%%    \centering
%%    \includegraphics[width=.9\linewidth]{logo_gips_unit.png}
%%    \caption{The logo of IPS, Waseda University.}
%%    \label{fig:example}
%%\end{figure}
%
%\begin{figure}[b!]
%\fcolorbox{wsdred}{wsdgrey}{ 
%%% first argument is the frame color
%%% second argument is the background color
%\begin{minipage}{.95\linewidth}
%\color{white}
%
%%% the title of the minipage
%\begin{center}
%    \fontsize{10}{12}\fontfamily{phv}\fontshape{sc}\selectfont
%    \textbf{About the School}
%\end{center}
%
%%\begin{wrapfigure}{r}{.3\linewidth}
%%\includegraphics[width=\linewidth]{logo_gips_unit.png}
%%\end{wrapfigure}
%\textbf{Waseda University Graduate School of Information, Production and Systems (IPS)} is a graduate school that has no corresponding undergraduate department established within the Kitakyushu Science and Research Park area in 2003 as the base for Waseda University to expand its presence in Asia. With three fields of study, including Information Architecture, Production System, and Integrated Systems, IPS undertakes academic research in the technological fields that society currently requires, and strives to attain a sustainable society through the use of technology. \footnotemark[1].
%\end{minipage}
%}
%\end{figure}
%
%\subsection{Tables}
%Creating tables is the same as any other latex documents, as shown in Table \ref{tab:example}.
%
%\begin{table}[t!]
%    \centering
%    \begin{tabular}{ccc}
%    \hline
%        header1 &  header2 & header 3\\
%        \hline
%        data1 & data2 & data3\\
%        data4 & data5 & data6
%    \end{tabular}
%    \caption{An example of a floating table.}
%    \label{tab:example}
%\end{table}
%
%\section{Equations}
%Just normally input your equations as either inline one: $E=h\frac{\lambda}{c}$ or with equation environment (equation* to remove numbering):
%
%\begin{equation}
%\left\{
%\begin{aligned}
%    &\oiint_{\partial\Omega}{\mathbf{E}\cdot d\mathbf{S}}=\frac 1{\varepsilon_0}\iiint_\Omega{\rho dV}\\
%    &\oiint_{\partial\Omega}{\mathbf{B}\cdot d\mathbf{S}}=0\\
%    &\oint_{\partial\Sigma}{\mathbf{E}\cdot d\boldsymbol{l}}=-\frac d{dt} \iint_\Sigma{{\mathbf{B}\cdot d\mathbf{S}}}\\
%    &\oint_{\partial\Sigma}{\mathbf{B}\cdot d\boldsymbol{l}}=\mu_0 \left(\iint_\Sigma{{\mathbf{J}\cdot d\mathbf{S}}}+\varepsilon_0\frac d{dt} \iint_\Sigma{{\mathbf{E}\cdot d\mathbf{S}}}\right)\cite{Maxwell1865}
%\end{aligned}
%\right.
%\end{equation}
%
%\appendices


\end{document}
